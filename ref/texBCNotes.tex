Three important attributes of blockchain are:

\begin{itemize}
\item[---] \textbf{Trust}: Blockchain helps in creating applications that are decentralized and collectively owned by multiple people. Nobody within this group has the power to change or delete previous transactions. Even if someone tries to do so, it will not be accepted by other stakeholders.
\item[---] \textbf{Autonomy} There is no single owner for blockchain-based applications. No one controls the blockchain, but everyone participates in its activities. This helps in creating solutions that cannot be manipulated or induce corruption.
\item[---] \textbf{Intermediaries} Blockchain-based applications can help remove the intermediaries from existing processes. Generally there is a central body, such as vehicle registration, license issuing, and so on, that acts as registrar for registering vehicles as well as issuing driver licenses. Without blockchain-based systems, there is no central body and if a license is issued or vehicle is registered after a blockchain mining process, that will remain a fact for an epoch time-period without the need of any central authority vouching for it.

\end{itemize}
There are the following two types of cryptography in computing: Symmetric and Asymmetric

\paragraph{Types of cryptography}
Symmetric cryptography refers to the process of using a single key for both encryption and decryption.

Asymmetric cryptography refers to the process of using two keys for encryption and decryption. Any key can be used for encryption and decryption. Message encryption with a public key can be decrypted using a private key and messages encrypted by a private key can be decrypted using a public key.


Hashing is the process of transforming any input data into fixed length random character data, and it is not possible to regenerate or identify the original data from the resultant string data. Hashes are also known as fingerprint of input data.

\paragraph{What is Gas}
This is also known as gas cost. There is also gas price that can be adjusted to a lower price when the price of Ether increases and a higher price when the price of Ether decreases. 

\section{Properties of Ethereum Transactions}
A transaction has some of the following important properties related to it:

\begin{itemize}
\item The from account property denotes the account that is originating the transaction and represents an account that is ready to send some gas or Ether. Both gas and Ether concepts were discussed earlier in this chapter. The from account can be externally owned or a contract account.
\item The to account property refers to an account that is receiving Ether or benefits in lieu of an exchange. For transactions related to deployment of contract, the to field is empty. It can be externally owned or a contract account.
\item The value account property refers to the amount of Ether that is transferred from one account to another.
\item The input account property refers to the compiled contract bytecode and is used during contract deployment in EVM. It is also used for storing data related to smart contract function calls along with its parameters. A typical transaction in Ethereum where a contract function is invoked is shown here. In the following screenshot, notice the input field containing the function call to contract along with its parameters:
\begin{itemize}
\item The blockHash account property refers to the hash of block to which this transaction belongs.
\item The blockNumber account property is the block in which this transaction belongs.
\item The gas account property refers to the amount of gas supplied by the sender who is executing this transaction.
\item The gasPrice account property refers to the price per gas the sender was willing to pay 
\item The hash account property refers to the hash of the transaction.
\item The nonce account property refers to the number of transactions made by the sender prior to the current transaction.
\item The nonce account property refers to the number of transactions made by the sender prior to the current transaction.
\item The value account property refers to the amount of Ether transferred in wei.
\item The v, r, and s account properties relate to digital signatures and the signing of the transaction.
\end{itemize}
\end{itemize}

\section{Properties of Ethereum Blocks}
\begin{itemize}
\item The difficulty property determines the complexity of the puzzle/challenge given to miners for this block.
\item The gasLimit property determines the maximum gas allowed. This helps in determining how many transactions can be part of the block.
\item The gasUsed property refers to the actual gas used for this block for executing all transactions in it.
\item The hash property refers to the hash of the block.
\item The nonce property refers to the number that helps in solving the challenge.
\item The miner property is the account identifier of the miner, also known as coinbase or etherbase.
\item The number property is the sequential number of this block on the chain.
\item The parentHash property refers to the parent block's hash.
\item The receiptsRoot, stateRoot, and transactionsRoot properties refer to Merkle trees discussed during the mining process.
\item The transactions property refers to an array of transactions that are part of this block.
\item The difficulty property determines the complexity of the puzzle/challenge given to miners for this block.
\end{itemize}
\paragraph{Mining Protocols}
\begin{description}
\item[Proof of Work] (PoW) wherein a miner provides proof that it is has worked on computing the final answer that could satisfy as solution to the puzzle.

\item[Proof of Stake (PoS)] concept states that a person can mine or validate block transactions according to how many coins he or she holds. The creator of the next block is chosen via various combinations of random selection and wealth or age (i.e., the stake)

\item[Proof of Authority (PoA)]  transactions and blocks are validated by approved accounts, known as validators. Validators run software allowing them to put transactions in blocks. The process is automated and does not require validators to be constantly monitoring their computers. It, however, does require maintaining the computer (the authority node) uncompromised.

\item[Proof of Space Time (PoST)] apparently able to run on any computer without expensive hardware. A PoST allows a prover to convince a verifier that she spent a "space-time" resource (storing data space over a period of time) \footnote{For PoST, see \url{https://eprint.iacr.org/2016/035.pdf}}
\end{description}

 Geth allows connectivity to JSON RPC using the following three different protocols:
\begin{itemize}
\item Inter Process Communication (IPC):  This protocol is used for inter process communication generally used within the same computer.
\item Remote Procedure Calls (RPC):  This protocol is used for inter process communication across computers. This is generally based on TCP and HTTP protocol.
\item Web Sockets (WS): This protocol is used to connect to Geth using sockets.
\end{itemize}

\begin{table}
\begin{tabular}{p{5cm} p{8cm}}
\hline 
 Variable Name & Description  \\ 
\hline 
block.blockhash(uint blockNumber) returns (bytes32) & hash of the given block - only works for 256 most recent, excluding current, blocks - deprecated in version 0.4.22 and replaced by blockhash(uint blockNumber).  \\ 
\hline 
 block.coinbase (address) & 
 current block miner's address \\ 
\hline 
 block.difficulty (uint): &  current block difficulty \\ 
\hline 
 block.gaslimit (uint) & current block gaslimit  \\ 
\hline 
 block.number (uint) & current block number \\ 
\hline 
 block.timestamp (uint) & current block timestamp as seconds since unix epoch \\ 
\hline 
gasleft() returns (uint256) & remaining gas  \\ 
\hline 
 msg.data (bytes) & complete calldata   \\ 
\hline 
 msg.gas (uint) to be gasLeft() & remaining gas - deprecated in version  \\ 
\hline 
msg.sender (address) & sender of the message (current call) \\ \hline
msg.sig (bytes4) & first four bytes of the calldata (i.e. function identifier) \\ \hline
msg.value (uint) & number of wei sent with the message \\ \hline 
now (uint) & current block timestamp (alias for block.timestamp) \\ \hline
tx.gasprice (uint) & gas price of the transaction \\ \hline 
tx.origin (address) & sender of the transaction (full call chain) \\ \hline   
\end{tabular} 
\end{table}


In solidity, there are a few ground rules for a library contract, which
differentiates it from a normal contract. These are:
\begin{itemize}
\item A library cannot have a state variable
\item A library cannot inherit or be inherited
\item A library cannot receive ether.
\end{itemize}

Solidity is a work-in-progress yet well-documented language, being
gradually developed as users face various problems while implementing
smart contracts on the blockchain.

\paragraph{Token vs Coin}
In simple words, the following three points summarize the difference
between coins and tokens:
\begin{itemize}
\item  Coins are separate currencies on their own blockchain, while
tokens are mainly based on a single blockchain variant
\item Coins generally have a limited functionality, store-of-value, while
tokens can store a complex, multi-faceted level of values.
\item Coins are mostly generated by mining a blockchain, while tokens
are generated by executing smart contracts on the blockchain.
\end{itemize}

\subsection{Ponzi Scheme}
\begin{itemize}
\item Array based pyramid scheme
\begin{itemize}
\item An early adopter in this scheme can redeem their multiplied investement when enough money is gathered from the useres who alter on join the scheme.
\item The last user must wait until all the users before her have redeemed their share
\end{itemize}
\item Tree based pyramid scheme
\begin{itemize}
\item Each user has a parent, called inviter, expect the root node, which is the contract owner
\item Whenever a user joins the scheme, their money is split among its ancestors
\item An user cannot foresee how much they will gain
\end{itemize}
\item Handover schemes
\begin{itemize}
\item Store only the address of the last user.
\item If someone wants to join, she must repay the last user of her investment plus a fixed interest.
\end{itemize}
\item Waterfall schemes
\begin{itemize}
\item Each new investment gets divided among the already-joined users, starting from the first one.
\item Each user receives a fixed percentage of what they have invested, as far as there is enough money.
\end{itemize}
\end{itemize}